% Options for packages loaded elsewhere
\PassOptionsToPackage{unicode}{hyperref}
\PassOptionsToPackage{hyphens}{url}
%
\documentclass[
]{article}
\usepackage{lmodern}
\usepackage{amssymb,amsmath}
\usepackage{ifxetex,ifluatex}
\ifnum 0\ifxetex 1\fi\ifluatex 1\fi=0 % if pdftex
  \usepackage[T1]{fontenc}
  \usepackage[utf8]{inputenc}
  \usepackage{textcomp} % provide euro and other symbols
\else % if luatex or xetex
  \usepackage{unicode-math}
  \defaultfontfeatures{Scale=MatchLowercase}
  \defaultfontfeatures[\rmfamily]{Ligatures=TeX,Scale=1}
\fi
% Use upquote if available, for straight quotes in verbatim environments
\IfFileExists{upquote.sty}{\usepackage{upquote}}{}
\IfFileExists{microtype.sty}{% use microtype if available
  \usepackage[]{microtype}
  \UseMicrotypeSet[protrusion]{basicmath} % disable protrusion for tt fonts
}{}
\makeatletter
\@ifundefined{KOMAClassName}{% if non-KOMA class
  \IfFileExists{parskip.sty}{%
    \usepackage{parskip}
  }{% else
    \setlength{\parindent}{0pt}
    \setlength{\parskip}{6pt plus 2pt minus 1pt}}
}{% if KOMA class
  \KOMAoptions{parskip=half}}
\makeatother
\usepackage{xcolor}
\IfFileExists{xurl.sty}{\usepackage{xurl}}{} % add URL line breaks if available
\IfFileExists{bookmark.sty}{\usepackage{bookmark}}{\usepackage{hyperref}}
\hypersetup{
  pdftitle={Energy Insecurity and Redlined America},
  pdfauthor={A. Justin Kirkpatrick},
  hidelinks,
  pdfcreator={LaTeX via pandoc}}
\urlstyle{same} % disable monospaced font for URLs
\usepackage[margin=1in]{geometry}
\usepackage{color}
\usepackage{fancyvrb}
\newcommand{\VerbBar}{|}
\newcommand{\VERB}{\Verb[commandchars=\\\{\}]}
\DefineVerbatimEnvironment{Highlighting}{Verbatim}{commandchars=\\\{\}}
% Add ',fontsize=\small' for more characters per line
\usepackage{framed}
\definecolor{shadecolor}{RGB}{248,248,248}
\newenvironment{Shaded}{\begin{snugshade}}{\end{snugshade}}
\newcommand{\AlertTok}[1]{\textcolor[rgb]{0.94,0.16,0.16}{#1}}
\newcommand{\AnnotationTok}[1]{\textcolor[rgb]{0.56,0.35,0.01}{\textbf{\textit{#1}}}}
\newcommand{\AttributeTok}[1]{\textcolor[rgb]{0.77,0.63,0.00}{#1}}
\newcommand{\BaseNTok}[1]{\textcolor[rgb]{0.00,0.00,0.81}{#1}}
\newcommand{\BuiltInTok}[1]{#1}
\newcommand{\CharTok}[1]{\textcolor[rgb]{0.31,0.60,0.02}{#1}}
\newcommand{\CommentTok}[1]{\textcolor[rgb]{0.56,0.35,0.01}{\textit{#1}}}
\newcommand{\CommentVarTok}[1]{\textcolor[rgb]{0.56,0.35,0.01}{\textbf{\textit{#1}}}}
\newcommand{\ConstantTok}[1]{\textcolor[rgb]{0.00,0.00,0.00}{#1}}
\newcommand{\ControlFlowTok}[1]{\textcolor[rgb]{0.13,0.29,0.53}{\textbf{#1}}}
\newcommand{\DataTypeTok}[1]{\textcolor[rgb]{0.13,0.29,0.53}{#1}}
\newcommand{\DecValTok}[1]{\textcolor[rgb]{0.00,0.00,0.81}{#1}}
\newcommand{\DocumentationTok}[1]{\textcolor[rgb]{0.56,0.35,0.01}{\textbf{\textit{#1}}}}
\newcommand{\ErrorTok}[1]{\textcolor[rgb]{0.64,0.00,0.00}{\textbf{#1}}}
\newcommand{\ExtensionTok}[1]{#1}
\newcommand{\FloatTok}[1]{\textcolor[rgb]{0.00,0.00,0.81}{#1}}
\newcommand{\FunctionTok}[1]{\textcolor[rgb]{0.00,0.00,0.00}{#1}}
\newcommand{\ImportTok}[1]{#1}
\newcommand{\InformationTok}[1]{\textcolor[rgb]{0.56,0.35,0.01}{\textbf{\textit{#1}}}}
\newcommand{\KeywordTok}[1]{\textcolor[rgb]{0.13,0.29,0.53}{\textbf{#1}}}
\newcommand{\NormalTok}[1]{#1}
\newcommand{\OperatorTok}[1]{\textcolor[rgb]{0.81,0.36,0.00}{\textbf{#1}}}
\newcommand{\OtherTok}[1]{\textcolor[rgb]{0.56,0.35,0.01}{#1}}
\newcommand{\PreprocessorTok}[1]{\textcolor[rgb]{0.56,0.35,0.01}{\textit{#1}}}
\newcommand{\RegionMarkerTok}[1]{#1}
\newcommand{\SpecialCharTok}[1]{\textcolor[rgb]{0.00,0.00,0.00}{#1}}
\newcommand{\SpecialStringTok}[1]{\textcolor[rgb]{0.31,0.60,0.02}{#1}}
\newcommand{\StringTok}[1]{\textcolor[rgb]{0.31,0.60,0.02}{#1}}
\newcommand{\VariableTok}[1]{\textcolor[rgb]{0.00,0.00,0.00}{#1}}
\newcommand{\VerbatimStringTok}[1]{\textcolor[rgb]{0.31,0.60,0.02}{#1}}
\newcommand{\WarningTok}[1]{\textcolor[rgb]{0.56,0.35,0.01}{\textbf{\textit{#1}}}}
\usepackage{graphicx,grffile}
\makeatletter
\def\maxwidth{\ifdim\Gin@nat@width>\linewidth\linewidth\else\Gin@nat@width\fi}
\def\maxheight{\ifdim\Gin@nat@height>\textheight\textheight\else\Gin@nat@height\fi}
\makeatother
% Scale images if necessary, so that they will not overflow the page
% margins by default, and it is still possible to overwrite the defaults
% using explicit options in \includegraphics[width, height, ...]{}
\setkeys{Gin}{width=\maxwidth,height=\maxheight,keepaspectratio}
% Set default figure placement to htbp
\makeatletter
\def\fps@figure{htbp}
\makeatother
\setlength{\emergencystretch}{3em} % prevent overfull lines
\providecommand{\tightlist}{%
  \setlength{\itemsep}{0pt}\setlength{\parskip}{0pt}}
\setcounter{secnumdepth}{-\maxdimen} % remove section numbering
\usepackage{bbm}
\usepackage{threeparttable}
\usepackage{float}
\usepackage{booktabs}

\title{Energy Insecurity and Redlined America}
\author{A. Justin Kirkpatrick}
\date{2021-01-21}

\begin{document}
\maketitle

This is a report generated by \texttt{knitr::spin()}

\hypertarget{introduction}{%
\section{Introduction}\label{introduction}}

This is the section with the introduction

\hypertarget{details-and-description}{%
\subsection{Details and description}\label{details-and-description}}

\hypertarget{define-energy-burden.}{%
\paragraph{Define energy burden.}\label{define-energy-burden.}}

\hypertarget{cite-severity-of-problem}{%
\paragraph{Cite severity of problem}\label{cite-severity-of-problem}}

\hypertarget{link-between-energy-burden-and-energy-efficiency}{%
\paragraph{Link between energy burden and energy
efficiency}\label{link-between-energy-burden-and-energy-efficiency}}

\hypertarget{obvious-reason-being-poor-and-short-sighted}{%
\paragraph{Obvious reason: being poor and
short-sighted}\label{obvious-reason-being-poor-and-short-sighted}}

\hypertarget{conditional-on-income-problem-still-persists}{%
\paragraph{Conditional on income, problem still
persists!}\label{conditional-on-income-problem-still-persists}}

\hypertarget{potential-explanations-brief}{%
\paragraph{Potential explanations
(brief)}\label{potential-explanations-brief}}

This section is brief. Full literature review comes later. Or maybe the
full lit for why is here?

\hypertarget{causal-chain-is-really-long}{%
\paragraph{Causal chain is really
long}\label{causal-chain-is-really-long}}

\hypertarget{potential-chain-redlining-lack-of-ownership-why-stay-in-area}{%
\paragraph{Potential chain (redlining--\textgreater{} lack of ownership
--\textgreater{} why stay in
area?)}\label{potential-chain-redlining-lack-of-ownership-why-stay-in-area}}

\hypertarget{this-paper-here}{%
\subsection{This paper here\ldots{}}\label{this-paper-here}}

\hypertarget{explores-the-role-of-redlining-in-explaining-energy-burden}{%
\paragraph{Explores the role of redlining in explaining energy
burden}\label{explores-the-role-of-redlining-in-explaining-energy-burden}}

By isolating the energy outcomes that can only be explained by redlining

\hypertarget{plausibly-exogenous-variation}{%
\paragraph{Plausibly exogenous
variation}\label{plausibly-exogenous-variation}}

Conditional on similar (but not red-lined) nearby neighborhoods.
Observables.

\hypertarget{extracted-1930s-data-to-understand-redlining}{%
\paragraph{Extracted 1930's data to understand
redlining}\label{extracted-1930s-data-to-understand-redlining}}

Survey data created by federal HOLC employees in 1936-1939 provides
neighborhood-specific data on observable characteristics that can be
used to identify the effects of redlining. The designation of red- and
yellow-lined areas (as well as green and blue) aggregate away important
variation between each grading. Not all redlined areas are identical,
nor are all yellow areas. Survey data recorded as part of the redline
designation process provides important, but largely unusued, sources of
variation. Within-grade variation in housing is common with reported
average rents, median income, repair status of housing, share of housing
developed, and construction type. Prior to being redlined, areas with
high percentages of low-income or minority populations were more likely
to have lower rents and lower housing quality. Frequently, the reason
for a low-income or minority area's location was associated with the
quality of the geography or proximity to natural features - low-lying
areas that frequently flooded or areas too steep for conventional
building were generally populated by lower-income individuals. These
same features also foment low investment in housing stock, and can
explain current inefficiencies in housing regardless of the HOLC grade.
With HOLC survey data in hand, I can control for these features and
separate out the effect of HOLC redlining from the determinants of HOLC
grading.

\hypertarget{identification-key-there-are-neighborhoods-with-identical-economics-and-racial-composition-where-some-surveyors-designated-them-red-and-some-designated-yellow.}{%
\paragraph{Identification key: there are neighborhoods with identical
economics and racial composition where some surveyors designated them
red and some designated
yellow.}\label{identification-key-there-are-neighborhoods-with-identical-economics-and-racial-composition-where-some-surveyors-designated-them-red-and-some-designated-yellow.}}

\hypertarget{poor-whites-vs.-poor-blacks}{%
\paragraph{Poor whites vs.~poor
blacks}\label{poor-whites-vs.-poor-blacks}}

\hypertarget{historic-data-merged-to-block-groups-to-leverage-modern-outcomes}{%
\paragraph{Historic data merged to block groups to leverage modern
outcomes}\label{historic-data-merged-to-block-groups-to-leverage-modern-outcomes}}

To assess current energy outcomes, I merge original HOLC neighborhoods
to modern census block groups, extracting ACS and census indicators of
high energy burden. Census block groups match the granularity of HOLC
neighborhoods reasonably well. However, boundaries do not tend to
coincide exactly, requiring some aggregation. Once linked, I compare
modern energy outcomes with HOLC grading, conditioning on observable
characteristics both in the 1930's (selection), and in the present.
While an individual household-level analysis would provide the clearest
evidence, household-level energy consumption data is limited for privacy
reasons, especially at the spatial resolution of HOLC neighborhoods,
which can have as few as 100 homes in them.

\hypertarget{section}{%
\paragraph{}\label{section}}

\hypertarget{data}{%
\section{Data}\label{data}}

This is the section with the data. And the section where we process the
data

\hypertarget{census-blockgroups}{%
\subsubsection{Census blockgroups}\label{census-blockgroups}}

HOLC grading and survey data are available at the neighborhood level,
where the average neighborhood size is approximately 0.72 {[}m\^{}2{]}
square kilometers.

\hypertarget{energy-burden}{%
\subsubsection{Energy burden}\label{energy-burden}}

I first examine the incidence of energy burden on minority populations.
A model of home selection would clearly predict a relationship between
energy burden and income as low-income individuals trade off energy
efficient (yet higher cost) housing for lower-cost but
energy-inefficient housing. Since a home's energy efficiency is part of
a bundle of attributes, households with low income may often trade
energy efficiency for other properties, such as larger square footage.
Lower efficiency homes may be more expensive to heat to a comfortable
standard. However, households with severe budget constraints may select
a low-efficiency house and select to spend as little as possible on
heating, resulting in very low inside temperatures, lower housing costs,
and low energy expenditures.

To show the relationship between home energy efficiency and income, I
regress the fraction of households reporting substandard heating
technology (coal, wood, fuel oil, or ``other''??) in a census block
group on measures of income using the following specification:

\begin{eqnarray*}
SubstandardHeating_{i} = \beta_0 + \beta_1 MedInc_i + \beta_2 xxx_i + \epsilon_i \\ \nonumber
\end{eqnarray*}

\begin{table}

\caption{\label{tab:unnamed-chunk-2}Share of Households with OtherSubstandard Fuel (Coal, wood, LP gas, other, and none)}
\centering
\resizebox{\linewidth}{!}{
\begin{tabular}[t]{lccc}
\toprule
  & Model 1 & Model 2 & Model 3\\
\midrule
MedIncome2018 & -0.00060** & -0.00019* & -0.00018\\
 & (0.00024) & (0.00011) & (0.00011)\\
Black & -0.12194 &  & \\
 & (0.12134) &  & \\
White & -0.11277 &  & \\
 & (0.11729) &  & \\
Asian & -0.10836 &  & \\
 & (0.12011) &  & \\
OtherRace & -0.06279 &  & \\
 & (0.06751) &  & \\
MedIncome2018 × Black & 0.00031 &  & \\
 & (0.00025) &  & \\
MedIncome2018 × White & 0.00047** &  & \\
 & (0.00022) &  & \\
\midrule
Num.Obs. & 32403 & 18573 & 13170\\
R2 & 0.245 & 0.267 & 0.287\\
R2 Adj. & 0.240 & 0.259 & 0.277\\
R2 Within & 0.044 & 0.017 & 0.015\\
R2 Pseudo &  &  & \\
AIC & -109852.6 & -59991.5 & -41706.4\\
BIC & -107999.3 & -58425.6 & -40269.1\\
Log.Lik. & 55147.296 & 30195.737 & 21045.196\\
FE: STCO & X & X & X\\
Std. errors & Clustered (STCO) & Clustered (STCO) & Clustered (STCO)\\
\bottomrule
\multicolumn{4}{l}{\textsuperscript{} * p < 0.1, ** p < 0.05, *** p < 0.01}\\
\multicolumn{4}{l}{\textsuperscript{} Robust SE clustered by FIPS county}\\
\end{tabular}}
\end{table}
\begin{table}

\caption{\label{tab:unnamed-chunk-3}Share of Households with OtherSubstandardMed Fuel (Coal, LP gas, Fuel Oil, other, and none)}
\centering
\resizebox{\linewidth}{!}{
\begin{tabular}[t]{lccc}
\toprule
  & Model 1 & Model 2 & Model 3\\
\midrule
MedIncome2018 & -0.00112*** & -0.00024* & -0.00024*\\
 & (0.00034) & (0.00013) & (0.00014)\\
Black & -0.20809* &  & \\
 & (0.11769) &  & \\
White & -0.20479* &  & \\
 & (0.11938) &  & \\
Asian & -0.18858* &  & \\
 & (0.11078) &  & \\
OtherRace & -0.07260 &  & \\
 & (0.06890) &  & \\
MedIncome2018 × Black & 0.00070** &  & \\
 & (0.00031) &  & \\
MedIncome2018 × White & 0.00118*** &  & \\
 & (0.00042) &  & \\
\midrule
Num.Obs. & 32403 & 18573 & 13170\\
R2 & 0.634 & 0.642 & 0.650\\
R2 Adj. & 0.632 & 0.639 & 0.645\\
R2 Within & 0.038 & 0.009 & 0.008\\
R2 Pseudo &  &  & \\
AIC & -72873.6 & -38849.0 & -26845.5\\
BIC & -71020.3 & -37283.1 & -25408.2\\
Log.Lik. & 36657.787 & 19624.511 & 13614.748\\
FE: STCO & X & X & X\\
Std. errors & Clustered (STCO) & Clustered (STCO) & Clustered (STCO)\\
\bottomrule
\multicolumn{4}{l}{\textsuperscript{} * p < 0.1, ** p < 0.05, *** p < 0.01}\\
\multicolumn{4}{l}{\textsuperscript{} Robust SE clustered by FIPS county}\\
\end{tabular}}
\end{table}

\hypertarget{results}{%
\paragraph{Results}\label{results}}

First table is using OtherSubstandard. Second is OtherSubstandardMed
(incl.~fuel oil)

Nothing good to report here on the first. High income, as expected,
means less likely to have substandard heating (median income; percent in
block-group).

\begin{table}[H]
\centering\begin{table}[H]
\centering
\resizebox{\linewidth}{!}{
\begin{tabular}[t]{lcccccccc}
\toprule
  & Model 1 & Model 2 & Model 3 & Model 4 & Model 5 & Model 6 & Model 7 & Model 8\\
\midrule
as.factor(GRADE.max)D & 0.00348** & 0.00338** & 0.00477** & 0.00383** & 0.00441*** & 0.00533*** & 0.00263 & 0.00515*\\
 & (0.00158) & (0.00159) & (0.00189) & (0.00155) & (0.00122) & (0.00181) & (0.00176) & (0.00302)\\
as.factor(GRADE.max)B & -0.00040 & -0.00032 & -0.00062 & -0.00031 & 0.00039 & -0.00066 & 0.00303 & 0.00777\\
 & (0.00112) & (0.00120) & (0.00147) & (0.00099) & (0.00146) & (0.00121) & (0.00385) & (0.01007)\\
as.factor(GRADE.max)A & 0.00025 & -0.00053 & -0.00571 & -0.00075 & -0.00004 & -0.00594 & -0.00414 & -0.02233\\
 & (0.00343) & (0.00338) & (0.00510) & (0.00332) & (0.00365) & (0.00504) & (0.00890) & (0.01409)\\
MedIncome2018 & -0.00009*** & -0.00010*** & -0.00011*** & -0.00012*** & -0.00012*** & -0.00013*** & -0.00017*** & -0.00023***\\
 & (0.00002) & (0.00002) & (0.00002) & (0.00002) & (0.00002) & (0.00002) & (0.00003) & (0.00005)\\
MedIncome1936 & 0.00026** & 0.00003 & 0.00006 & 0.00009 & 0.00007 & 0.00012 & -0.00014 & -0.00024\\
 & (0.00011) & (0.00014) & (0.00014) & (0.00014) & (0.00015) & (0.00015) & (0.00066) & (0.00107)\\
Rent3739\_Mean & 0.00007* & 0.00007* & 0.00005 & 0.00006* & 0.00006* & 0.00005 & 0.00003 & 0.00005\\
 & (0.00004) & (0.00003) & (0.00005) & (0.00003) & (0.00003) & (0.00005) & (0.00002) & (0.00005)\\
MedIncome2018 × MedIncome1936 &  & 0.00000** & 0.00000*** & 0.00000*** & 0.00000** & 0.00000** & 0.00001 & 0.00003\\
 &  & (0.00000) & (0.00000) & (0.00000) & (0.00000) & (0.00000) & (0.00001) & (0.00002)\\
Black &  &  &  & -0.00219 & -0.00115 & -0.00104 & -0.00541 & -0.00688\\
 &  &  &  & (0.00455) & (0.00577) & (0.00481) & (0.00930) & (0.01490)\\
White &  &  &  & 0.00304 & 0.00303 & 0.00464 & 0.00648 & 0.00890\\
 &  &  &  & (0.00436) & (0.00454) & (0.00527) & (0.00805) & (0.01295)\\
Asian &  &  &  & -0.00011 & 0.00001 & 0.00108 & -0.00326 & -0.00356\\
 &  &  &  & (0.00451) & (0.00490) & (0.00523) & (0.01133) & (0.01692)\\
as.factor(GRADE.max)D × Black &  &  &  &  & -0.00199 &  &  & \\
 &  &  &  &  & (0.00411) &  &  & \\
as.factor(GRADE.max)B × Black &  &  &  &  & -0.00285 &  &  & \\
 &  &  &  &  & (0.00273) &  &  & \\
as.factor(GRADE.max)A × Black &  &  &  &  & -0.00573 &  &  & \\
 &  &  &  &  & (0.01121) &  &  & \\
NBlack\_YN &  &  &  &  &  &  & -0.00211 & 0.00131\\
 &  &  &  &  &  &  & (0.00316) & (0.00371)\\
NBlack\_PCT &  &  &  &  &  &  & -0.00000 & -0.00002\\
 &  &  &  &  &  &  & (0.00004) & (0.00005)\\
\midrule
Num.Obs. & 4629 & 4629 & 3313 & 4629 & 4629 & 3313 & 1034 & 690\\
R2 & 0.097 & 0.098 & 0.097 & 0.101 & 0.101 & 0.100 & 0.100 & 0.111\\
R2 Adj. & 0.086 & 0.087 & 0.081 & 0.089 & 0.089 & 0.084 & 0.052 & 0.049\\
R2 Within & 0.021 & 0.021 & 0.026 & 0.025 & 0.025 & 0.030 & 0.032 & 0.044\\
R2 Pseudo &  &  &  &  &  &  &  & \\
AIC & -22109.1 & -22110.3 & -15583.4 & -22120.6 & -22115.8 & -15590.5 & -4692.8 & -3006.6\\
BIC & -21742.0 & -21736.8 & -15235.4 & -21727.7 & -21703.7 & -15224.2 & -4426.0 & -2797.9\\
Log.Lik. & 11111.557 & 11113.148 & 7848.704 & 11121.281 & 11121.917 & 7855.252 & 2400.393 & 1549.309\\
FE: STCO & X & X & X & X & X & X & X & X\\
Std. errors & Clustered (STCO) & Clustered (STCO) & Clustered (STCO) & Clustered (STCO) & Clustered (STCO) & Clustered (STCO) & Clustered (STCO) & Clustered (STCO)\\
\bottomrule
\multicolumn{9}{l}{\textsuperscript{} * p < 0.1, ** p < 0.05, *** p < 0.01}\\
\multicolumn{9}{l}{\textsuperscript{} Robust SE clustered by FIPS county}\\
\end{tabular}}
\end{table}
\end{table}

\hypertarget{results-1}{%
\paragraph{Results}\label{results-1}}

Columns 1, 2, 3, and 5 show results using all blockgroups with greater
than 80\% of total area contained within one HOLC grade. Columns 4, 6,
and 8 set the threshold at 95\%. Column 1 allows an additive effect for
median income in 2018 and 1936 (reported from HOLC surveys), while 2-6
allow an interaction.

As expected, current median blockgroup income predicts a lower share of
homes with substandard heating fuel across all models. Median income in
1936 predicts \textbf{higher} prevalance of substandard heating in
higher 1936 median income areas. Can I explain this?

Columns 4-7 include controls for current racial composition. These
results consistently find that higher percentages of Blacks are
associated with lower likelihood of substandard heating fuel.
\textbf{EXPLAIN!!}

The coefficient on HOLC Grade D is positive and significant across all
specifications, indicating that, conditional on a rich set of controls
including 1936 characteristics to control for selection, areas that were
graded ``D'' by the HOLC are more likely to have substandard heating
fuels in 2018 relative to those graded ``C''.

\hypertarget{results-using-zip-code-tabulation-area-zcta}{%
\subsubsection{Results using Zip Code Tabulation Area
(ZCTA)}\label{results-using-zip-code-tabulation-area-zcta}}

Zipcode Tabulation Areas are coarser than census blockgroups. Due to
this, there are fewer ZCTAs that fall predominantly in one HOLC grade,
resulting in a smaller sample size. Of the 2,842 zip codes that touch on
one or more HOLC neighborhoods, 45 zip codes have greater than 80\%
within one HOLC grade. Of these, 19 are Grade D (red). Of these, only 2
zip code(s) have HOLC survey data including median income, presence of
minorities, and rent data for 1936-38. This precludes the use of zip
code aggregations to estimate HOLC grading on current substandard
heating fuel.

\hypertarget{household-level-energy}{%
\subsubsection{Household-level energy}\label{household-level-energy}}

Aggregation can mask important heterogeneity in household energy
consumption. We use household-level monthly consumption data geolocated
to the zip-code level to identify HOLC-zone specific energy consumption
responses to cooling and heating events.

Household energy consumption data is retrieved from the California
Residential Appliance Saturation Survey (RASS) for 2009 and (hopefully)
2019. The RASS is commissioned by the California Public Utilities
Commission and implemented by DNV GL Energy Insights. The survey
contains information on household energy consumption including home age,
primary heating fuel source, and thermostat setpoint. The survey also
includes the household's zip code, household characteristics including
income and number of children, and merges two years of billing
information obtained direclty from the gas and electric utilities
serving the household. The survey sample consists of 25721 households
sampled from all over California.

We are interested in household's energy consumption response to
increases in heating degree days. We focus on households that rely
primarily on electric heating. For each household, we estimate a
consumption response function that summarizes the household's change in
electricity consumption per change in monthly heating degree days. This
measure will be larger if a household consumes more energy when
temperatures are lower, and smaller if a household consumes less energy
when temperatures drop. I allow this response to vary bsed on the HOLC
grade that covers the plurality of the zip code in which the household
lies. I use only those zip codes in which greater than 80\% of the zip
code is within one specific HOLC grade.

\emph{A priori}, it is not clear whether the interaction of heating
degree days and HOLC-designated redlining should be positive or
negative. If a household prefers to remain warm and comfortable on a
cold night, then expenditures will be higher when temperatures are
lower. Similarly, if a household in a redlined area maintains the same
indoor temperature setpoint but has a home with lower energy ratings or
is otherwise less efficient, then energy expenditures will be greater as
well. If expenditures are not greater, then it may be that the household
trades off comfort by lowering the indoor temperature in order to keep
expenditures low, or it may be that the home is very efficient, and more
energy is not needed to maintain a constant temperature. This ambiguity
confounds interpretation of estimates.

Table (below) reports the results from a regression of the form:

\[hdd^e_h = \beta_0 + \sum 1(grade_h=g)\beta_g + \beta_{inc} avgincome_h + cdd^e_h + \gamma^{CZ} + \varepsilon\]

Where \(hdd^e\) is the electricity consumption response to one
additional heating degree day for household \(h\) located in HOLC grade
\(g\). \(cdd^e\) is the household's electricity consumption response to
an additional cooling degree day. Table (below that) shows results for
\(hdd^g\). \(\gamma^{CZ}\) are climate-zone fixed effects.

\begin{table}[H]
\centering
\begin{tabular}[t]{lcc}
\toprule
  & Model 1 & Model 2\\
\midrule
as.factor(GRADE.max)D & 0.290*** & -0.057\\
 & (0.049) & (0.042)\\
as.factor(GRADE.max)X & 0.227*** & 0.240***\\
 & (0.029) & (0.040)\\
avginc & 0.000 & 0.000\\
 & (0.000) & (0.000)\\
cdd.e &  & 0.002\\
 &  & (0.003)\\
\midrule
Num.Obs. & 14822 & 9220\\
R2 & 0.107 & 0.109\\
R2 Adj. & 0.107 & 0.108\\
R2 Within & 0.004 & 0.007\\
R2 Pseudo &  & \\
AIC & 13135.9 & 7592.6\\
BIC & 13227.2 & 7685.3\\
Log.Lik. & -6555.968 & -3783.290\\
FE: CZT24 & X & X\\
Std. errors & Clustered (CZT24) & Clustered (CZT24)\\
\bottomrule
\multicolumn{3}{l}{\textsuperscript{} * p < 0.1, ** p < 0.05, *** p < 0.01}\\
\end{tabular}
\end{table}

\begin{table}[H]
\centering
\begin{tabular}[t]{lcc}
\toprule
  & Model 1 & Model 2\\
\midrule
as.factor(GRADE.max)D & -4.591*** & -3.988***\\
 & (1.194) & (0.740)\\
avginc & 0.000 & 0.000\\
 & (0.000) & (0.000)\\
cdd.e &  & 0.720***\\
 &  & (0.176)\\
\midrule
Num.Obs. & 988 & 988\\
R2 & 0.206 & 0.268\\
R2 Adj. & 0.198 & 0.260\\
R2 Within & 0.068 & 0.141\\
R2 Pseudo &  & \\
AIC & 7034.5 & 6956.2\\
BIC & 7088.3 & 7015.0\\
Log.Lik. & -3506.238 & -3466.109\\
FE: CZT24 & X & X\\
Std. errors & Clustered (CZT24) & Clustered (CZT24)\\
\bottomrule
\multicolumn{3}{l}{\textsuperscript{} * p < 0.1, ** p < 0.05, *** p < 0.01}\\
\end{tabular}
\end{table}

Coefficient results in Model (1) indicate that households located inside
the HOLC red (D) areas have significantly higher gas responses, but
significantly lower electricity responses. In each case, only households
that report primary heating fuel of gas (first table) and electricity
(second table) are included. The ambiguity in effect is clear in
examining Table 2, which shows a significantly \emph{lower} effect
within the HOLC red (D) areas.

To address this confounding, I leverage the household's response to the
question on thermostat setpoint. Conditioning on a specific overnight or
daytime thermostat setting forecloses the possibility that households
adjust thermostat setting since the adjustment is answered in the
question. Household consumption response functions for those that save
on heating by setting the thermostat setpoint very low are identified by
variation conditional on their setpoint. By allowing each setpoint bin
(\textless55, 55-60, 60-65, 65-70, 70-75, 75+) to have a separate
estimate of response to heating degree days, I capture the effect of
heating degree days separate from thermostat setpoint adjustments. Once
properly conditioned, the difference between HOLC red and HOLC yellow
housing can be estimated.

\[usage^g_h = \beta_0 + \beta_1 hdd_h + \beta_2 hdd_h*(grade_h=D) + \beta_3 averageincome_h*hdd_h + \beta_4 cdd^e_h*hdd_h + \sum_{s=1}^{S} hdd_h*\theta^s + \kappa_h + \varepsilon\]

Where \(usage^g_h\) is the monthly observed gas usage, \(hdd_h\) is the
household's monthly heating degree days, \(s\in S\) are the temperature
setpoint bins, and \(\kappa_h\) is a vector of household fixed effects.

\begin{Shaded}
\begin{Highlighting}[]
\CommentTok{# What if we pool all of the monthly bill observations? Turns out, there are very few (if any) NG-using Hh's in D!  dplyr::filter(cdd<=10 & (PHTNGCNT==1|PHTNGRAD==1))}
\NormalTok{all09.use.pooled =}\StringTok{ }\NormalTok{all09.use }\OperatorTok\StringTok{ }\KeywordTok{unnest}\NormalTok{(edata) }\OperatorTok\StringTok{ }\NormalTok{dplyr}\OperatorTok{::}\KeywordTok{filter}\NormalTok{(share.max }\OperatorTok{>}\StringTok{ }\FloatTok{.90} \OperatorTok{&}\StringTok{ }\NormalTok{cdd}\OperatorTok{<=}\DecValTok{0} \OperatorTok{&}\StringTok{ }\NormalTok{(PHTELCRH}\OperatorTok{==}\DecValTok{1}\OperatorTok{|}\NormalTok{PHTELBSB}\OperatorTok{==}\DecValTok{1}\NormalTok{)) }\OperatorTok
\StringTok{  }\NormalTok{dplyr}\OperatorTok{::}\KeywordTok{mutate}\NormalTok{(}\DataTypeTok{NITESET =} \KeywordTok{relevel}\NormalTok{(NITESET, }\DataTypeTok{ref =} \StringTok{'>75'}\NormalTok{),}
                \DataTypeTok{DAYSET =} \KeywordTok{relevel}\NormalTok{(DAYSET, }\DataTypeTok{ref =} \StringTok{'>75'}\NormalTok{),}
                \DataTypeTok{ethnicity =} \KeywordTok{relevel}\NormalTok{(ethnicity, }\DataTypeTok{ref =} \StringTok{'White'}\NormalTok{),}
                \DataTypeTok{avginc1000 =}\NormalTok{ avginc}\OperatorTok{/}\DecValTok{1000}\NormalTok{)}

\KeywordTok{modelsummary}\NormalTok{(}\KeywordTok{list}\NormalTok{(}
  \KeywordTok{feols}\NormalTok{(u }\OperatorTok{~}\StringTok{ }\NormalTok{hdd }\OperatorTok{+}\StringTok{ }\NormalTok{hdd}\OperatorTok{:}\NormalTok{GRADE.max }\OperatorTok{+}\StringTok{ }\NormalTok{hdd}\OperatorTok{:}\NormalTok{avginc }\OperatorTok{|}\StringTok{ }\NormalTok{IDENT, }\DataTypeTok{weights=}\OperatorTok{~}\NormalTok{d, }\DataTypeTok{data =}\NormalTok{ all09.use.pooled, }\DataTypeTok{warn =}\NormalTok{ F), }\CommentTok{# Do redlined areas have diff. responses?}
  \KeywordTok{feols}\NormalTok{(u }\OperatorTok{~}\StringTok{ }\NormalTok{hdd }\OperatorTok{+}\StringTok{ }\NormalTok{hdd}\OperatorTok{:}\NormalTok{GRADE.max }\OperatorTok{+}\StringTok{ }\NormalTok{hdd}\OperatorTok{:}\NormalTok{avginc }\OperatorTok{+}\StringTok{ }\NormalTok{hdd}\OperatorTok{:}\NormalTok{cdd.e }\OperatorTok{|}\StringTok{ }\NormalTok{IDENT, }\DataTypeTok{weights=}\OperatorTok{~}\NormalTok{d, }\DataTypeTok{data =}\NormalTok{ all09.use.pooled, }\DataTypeTok{warn =}\NormalTok{ F) , }\CommentTok{# Do redlined areas have diff. responses controlling for building envelope (w/AC usage)}
  \KeywordTok{feols}\NormalTok{(u }\OperatorTok{~}\StringTok{ }\NormalTok{hdd }\OperatorTok{+}\StringTok{ }\NormalTok{hdd}\OperatorTok{:}\NormalTok{NITESET }\OperatorTok{+}\StringTok{ }\NormalTok{hdd}\OperatorTok{:}\NormalTok{GRADE.max }\OperatorTok{+}\StringTok{ }\NormalTok{hdd}\OperatorTok{:}\NormalTok{avginc }\OperatorTok{+}\StringTok{ }\NormalTok{hdd}\OperatorTok{:}\NormalTok{cdd.e }\OperatorTok{|}\StringTok{ }\NormalTok{IDENT, }\DataTypeTok{weights=}\OperatorTok{~}\NormalTok{d, }\DataTypeTok{data =}\NormalTok{ all09.use.pooled, }\DataTypeTok{warn =}\NormalTok{ F) , }\CommentTok{# Do redlined areas have diff. responses cond. on setpoint}
  \KeywordTok{feols}\NormalTok{(u }\OperatorTok{~}\StringTok{ }\NormalTok{hdd }\OperatorTok{+}\StringTok{ }\NormalTok{hdd}\OperatorTok{:}\NormalTok{DAYSET }\OperatorTok{+}\StringTok{ }\NormalTok{hdd}\OperatorTok{:}\NormalTok{GRADE.max }\OperatorTok{+}\StringTok{ }\NormalTok{hdd}\OperatorTok{:}\NormalTok{avginc }\OperatorTok{+}\StringTok{ }\NormalTok{hdd}\OperatorTok{:}\NormalTok{cdd.e }\OperatorTok{|}\StringTok{ }\NormalTok{IDENT, }\DataTypeTok{weights=}\OperatorTok{~}\NormalTok{d, }\DataTypeTok{data =}\NormalTok{ all09.use.pooled, }\DataTypeTok{warn =}\NormalTok{ F) ),}\CommentTok{# Or daytime setting}
  \DataTypeTok{stars =}\NormalTok{ T)}
\end{Highlighting}
\end{Shaded}

\begin{table}[H]
\centering
\begin{tabular}[t]{lcccc}
\toprule
  & Model 1 & Model 2 & Model 3 & Model 4\\
\midrule
hdd & 1.083*** & 1.068*** & 2.700*** & 2.644***\\
 & (0.264) & (0.278) & (0.205) & (0.201)\\
hdd × GRADE.maxD & 3.729*** & 3.821*** & 2.917*** & 3.445***\\
 & (0.407) & (0.407) & (0.728) & (0.614)\\
hdd × avginc & 0.000 & 0.000 & 0.000 & 0.000\\
 & (0.000) & (0.000) & (0.000) & (0.000)\\
hdd × cdd.e &  & -0.049 & -0.058* & -0.078*\\
 &  & (0.036) & (0.031) & (0.044)\\
hdd × NITESETOff &  &  & -1.348*** & \\
 &  &  & (0.454) & \\
hdd × NITESET55-60 &  &  & -1.418* & \\
 &  &  & (0.769) & \\
hdd × NITESET60-65 &  &  & -2.664*** & \\
 &  &  & (0.693) & \\
hdd × NITESET65-70 &  &  & -0.604 & \\
 &  &  & (0.536) & \\
hdd × NITESET70-75 &  &  & -1.793*** & \\
 &  &  & (0.539) & \\
hdd × NITESETUnk &  &  & -1.921*** & \\
 &  &  & (0.163) & \\
hdd × DAYSETOff &  &  &  & -1.038**\\
 &  &  &  & (0.508)\\
hdd × DAYSET55-60 &  &  &  & -2.035***\\
 &  &  &  & (0.404)\\
hdd × DAYSET60-65 &  &  &  & -2.614***\\
 &  &  &  & (0.891)\\
hdd × DAYSET65-70 &  &  &  & -1.333**\\
 &  &  &  & (0.573)\\
hdd × DAYSET70-75 &  &  &  & -1.421**\\
 &  &  &  & (0.599)\\
hdd × DAYSETUnk &  &  &  & -1.978***\\
 &  &  &  & (0.204)\\
\midrule
Num.Obs. & 1150 & 476 & 476 & 476\\
R2 & 0.846 & 0.902 & 0.907 & 0.906\\
R2 Adj. & 0.824 & 0.877 & 0.881 & 0.880\\
R2 Within & 0.089 & 0.250 & 0.285 & 0.280\\
R2 Pseudo &  &  &  & \\
AIC & 15451.3 & 6323.9 & 6312.9 & 6316.5\\
BIC & 16163.0 & 6732.1 & 6746.1 & 6749.7\\
Log.Lik. & -7584.656 & -3063.967 & -3052.463 & -3054.233\\
FE: IDENT & X & X & X & X\\
Std. errors & Clustered (IDENT) & Clustered (IDENT) & Clustered (IDENT) & Clustered (IDENT)\\
\bottomrule
\multicolumn{5}{l}{\textsuperscript{} * p < 0.1, ** p < 0.05, *** p < 0.01}\\
\end{tabular}
\end{table}

Results have the expected sign - an increase in the heating degree days
leads to an increase in gas consumption across each of the
specifications. Households located inside a HOLC red (D) area show
around two to three times the consumption per hdd relative to the
omitteed category, Grade C. Unfortunately, too few households lie in
HOLC areas with reported covariates necessary to control for
unobservables that may have affected the treatment (assignment to HOLC
red) and the current outcome (home efficiency / gas consumption per
hdd). Column (3) and (4) control for the reported nighttime and daytime
temperature setpoints. In both 3 and 4, the main effect remains (and
increases in magnitude) - conditional on a target setpoint, an increase
in \(hdd\) leads to an increase in usage. Although insignificant, the
interaction for the two lowest bins, 55-60 and 60-65, is negative,
indicating that an increase in \(hdd\) for households with very low
overnight setpoints leads to smaller increases in gas consumption
relative to the omitted category, which is ``off/other''. As expected,
households with very high overnight setpoints (\textgreater75) have very
high consumption responses to \(hdd\).

Notably, households within the HOLC red (D) area continue to exhibit two
to three times the consumption response relative to households in the
omitted category, even accounting for potentially heterogeneous
preferences for overnight temperature setpoints. This indicates that
households in HOLC red areas are not simply consuming more energy or
exhibiting greater preference for overnight comfort, but rather face
higher expenditures simply to maintain one constant temperature.
\#\#\#\# Household level energy by ethnicity This section examines the
coefficient of response across ethnicity, as well as HOLC grade and
thermostat set points.

\begin{table}[H]
\centering
\begin{tabular}[t]{lcccc}
\toprule
  & Model 1 & Model 2 & Model 3 & Model 4\\
\midrule
hdd & 1.601*** & 1.188*** & 1.585*** & 1.478***\\
 & (0.402) & (0.309) & (0.347) & (0.464)\\
hdd × GRADE.maxD & 1.764** & 1.120 & 1.770** & 1.351\\
 & (0.733) & (1.016) & (0.807) & (1.746)\\
hdd × ethnicityAsian and Pacific Islander & -0.674* & -0.467 & -0.948* & -0.782\\
 & (0.370) & (0.311) & (0.491) & (0.767)\\
hdd × ethnicityBlack & -0.869** & -0.460* & -0.982** & -0.708\\
 & (0.405) & (0.270) & (0.387) & (0.476)\\
hdd × ethnicityHispanic & -0.093 & 0.211 & -0.392 & -0.435\\
 & (0.546) & (0.511) & (0.583) & (0.943)\\
hdd × ethnicityOther & 2.066*** & 1.905*** & 1.755** & 1.895\\
 & (0.697) & (0.719) & (0.862) & (1.617)\\
hdd × ethnicityMixed & 2.119*** & 2.666*** & 1.848*** & 2.972\\
 & (0.383) & (0.662) & (0.566) & (2.169)\\
hdd × avginc1000 & -0.002 & -0.001 & -0.001 & -0.001\\
 & (0.003) & (0.003) & (0.003) & (0.004)\\
hdd × NITESETOff &  & 0.563 &  & 4.544**\\
 &  & (0.465) &  & (1.730)\\
hdd × NITESET55-60 &  & 0.256 &  & 1.259\\
 &  & (0.757) &  & (1.701)\\
hdd × NITESET60-65 &  & -0.978 &  & 1.794\\
 &  & (0.742) &  & (1.725)\\
hdd × NITESET65-70 &  & 1.182** &  & 2.471\\
 &  & (0.587) &  & (1.913)\\
hdd × NITESET70-75 &  & -0.160 &  & 2.687\\
 &  & (0.497) &  & (2.410)\\
hdd × DAYSETOff &  &  & 0.722 & -2.292\\
 &  &  & (0.533) & (1.536)\\
hdd × DAYSET55-60 &  &  & -0.697 & -2.657\\
 &  &  & (0.524) & (2.397)\\
hdd × DAYSET60-65 &  &  & -1.239 & -4.875***\\
 &  &  & (1.044) & (1.808)\\
hdd × DAYSET65-70 &  &  & 0.199 & -2.311\\
 &  &  & (0.675) & (1.485)\\
hdd × DAYSET70-75 &  &  & 0.248 & -3.262\\
 &  &  & (0.575) & (2.613)\\
\midrule
Num.Obs. & 1086 & 1086 & 1086 & 1086\\
R2 & 0.846 & 0.848 & 0.848 & 0.837\\
R2 Adj. & 0.824 & 0.825 & 0.825 & 0.811\\
R2 Within & 0.095 & 0.104 & 0.107 & 0.042\\
R2 Pseudo &  &  &  & \\
AIC & 14625.9 & 14624.4 & 14621.6 & 14707.4\\
BIC & 15314.6 & 15338.0 & 15335.2 & 15446.0\\
Log.Lik. & -7174.967 & -7169.180 & -7167.811 & -7205.697\\
FE: IDENT & X & X & X & X\\
Std. errors & Clustered (IDENT) & Clustered (IDENT) & Clustered (IDENT) & Clustered (IDENT)\\
\bottomrule
\multicolumn{5}{l}{\textsuperscript{} * p < 0.1, ** p < 0.05, *** p < 0.01}\\
\end{tabular}
\end{table}

Results surprisingly show that Black households with natural gas as a
primary heating fuel have lower responses to increases in heating degree
days relative to White households. A similar effect holds for Hispanic
households, though in no specifications are the results statistically
significant for Hispanic households. Conditioning on separate effects
for nighttime and daytime thermostat setpoints reduces the magnitude of
the coefficient of response for households in HOLC grade red (D), but
the point estimate is still positive.

\begin{table}[H]
\centering
\begin{tabular}[t]{lcccc}
\toprule
  & Model 1 & Model 2 & Model 3 & Model 4\\
\midrule
hdd & 0.326*** & 0.307*** & 0.331*** & 0.308***\\
 & (0.082) & (0.088) & (0.090) & (0.088)\\
hdd × GRADE.maxX & -0.093 & -0.110 & -0.122* & -0.118*\\
 & (0.072) & (0.073) & (0.071) & (0.071)\\
hdd × ethnicityNative American & -0.037 & -0.077 & -0.024 & -0.051\\
 & (0.053) & (0.074) & (0.054) & (0.063)\\
hdd × ethnicityAsian and Pacific Islander & -0.077*** & -0.066*** & -0.061** & -0.054**\\
 & (0.029) & (0.025) & (0.027) & (0.026)\\
hdd × ethnicityBlack & -0.069* & -0.056 & -0.054 & -0.038\\
 & (0.038) & (0.034) & (0.033) & (0.031)\\
hdd × ethnicityHispanic & -0.080*** & -0.063*** & -0.057** & -0.050**\\
 & (0.028) & (0.022) & (0.026) & (0.024)\\
hdd × ethnicityOther & 0.073 & 0.069 & 0.052 & 0.049\\
 & (0.056) & (0.048) & (0.063) & (0.058)\\
hdd × ethnicityMixed & -0.083** & -0.086* & -0.061* & -0.052\\
 & (0.039) & (0.052) & (0.033) & (0.036)\\
hdd × avginc1000 & 0.000 & 0.000 & 0.000 & 0.000\\
 & (0.000) & (0.000) & (0.000) & (0.000)\\
hdd × NITESETOff &  & 0.021 &  & 0.040\\
 &  & (0.035) &  & (0.046)\\
hdd × NITESET<55 &  & 0.007 &  & 0.068\\
 &  & (0.053) &  & (0.071)\\
hdd × NITESET55-60 &  & 0.112 &  & 0.145\\
 &  & (0.090) &  & (0.115)\\
hdd × NITESET60-65 &  & 0.043 &  & 0.054\\
 &  & (0.037) &  & (0.053)\\
hdd × NITESET65-70 &  & 0.052 &  & 0.045\\
 &  & (0.039) &  & (0.048)\\
hdd × NITESET70-75 &  & 0.028 &  & 0.032\\
 &  & (0.039) &  & (0.052)\\
hdd × NITESETUnk &  & -0.023 &  & -0.028\\
 &  & (0.047) &  & (0.052)\\
hdd × DAYSETOff &  &  & -0.014 & -0.035\\
 &  &  & (0.039) & (0.042)\\
hdd × DAYSET<55 &  &  & -0.063 & -0.131*\\
 &  &  & (0.061) & (0.072)\\
hdd × DAYSET55-60 &  &  & 0.013 & -0.062\\
 &  &  & (0.045) & (0.072)\\
hdd × DAYSET60-65 &  &  & 0.017 & -0.031\\
 &  &  & (0.039) & (0.052)\\
hdd × DAYSET65-70 &  &  & 0.069 & 0.038\\
 &  &  & (0.055) & (0.051)\\
hdd × DAYSET70-75 &  &  & 0.050 & 0.027\\
 &  &  & (0.041) & (0.048)\\
hdd × DAYSETUnk &  &  & -0.046 & \\
 &  &  & (0.051) & \\
\midrule
Num.Obs. & 3564 & 3564 & 3564 & 3564\\
R2 & 0.954 & 0.954 & 0.954 & 0.955\\
R2 Adj. & 0.944 & 0.944 & 0.944 & 0.944\\
R2 Within & 0.143 & 0.146 & 0.147 & 0.150\\
R2 Pseudo &  &  &  & \\
AIC & 38783.9 & 38786.0 & 38781.6 & 38782.2\\
BIC & 42806.2 & 42851.6 & 42847.2 & 42884.9\\
Log.Lik. & -18740.952 & -18735.022 & -18732.813 & -18727.119\\
FE: IDENT & X & X & X & X\\
Std. errors & Clustered (IDENT) & Clustered (IDENT) & Clustered (IDENT) & Clustered (IDENT)\\
\bottomrule
\multicolumn{5}{l}{\textsuperscript{} * p < 0.1, ** p < 0.05, *** p < 0.01}\\
\end{tabular}
\end{table}

Here, an insufficient number of gas households in HOLC grade D (red)
areas precludes us from seeing the effect of HOLC red on response to
heating degree days. Results focused on ethnicity show that Hispanic
households are less responsive to heating degree days conditional on
daytime and nighttime thermostat setpoints. A similar result for Black
households is not significant. Table blah blah blah \# End

\begin{Shaded}
\begin{Highlighting}[]
\CommentTok{###################}
\CommentTok{###################}
\CommentTok{###################}
\end{Highlighting}
\end{Shaded}

\end{document}
